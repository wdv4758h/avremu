
% We define 2 IO Registers to communicate from AVR to the Drawing
% mechanism

\newcount\avr@draw@x
\newcount\avr@draw@y
\newcount\avr@draw@maxx
\newcount\avr@draw@maxy

\def\avr@draw@init{%
  \avr@draw@maxx = 0%
  \avr@draw@maxy = 0%
}


% TWAR
\newcount\avr@draw@argc
\csdef{avr@io@000010@set}#1{%
  \avr@bin@tocount{#1}{\avr@accA}%
  \avr@log{DRAW CMD: \the\avr@accA}%
  \avr@draw@command{\the\avr@accA}%
  \avr@draw@argc = 0\relax
}

\csdef{avr@io@000011@set}#1{%
  \avr@bin@tocount{#1}{\avr@accA}%
  \avr@debug{AVR DRAW PUSH: \the\avr@accA}%
  \csxdef{avr@draw@stack@\the\avr@draw@argc}{\the\avr@accA}%
  \advance\avr@draw@argc by 1\relax
}

\def\avr@draw@stack#1{\csuse{avr@draw@stack@#1}}


\def\avr@draw@command#1{%
  \ifcsdef{avr@draw@command@#1}{%
    \csuse{avr@draw@command@#1}%
  }{%
    \avr@error{AVR DRAW: Unkown Command #1}%
  }%
}


%%%%%%%%%%%%%%%%%%%%%%%%%%%%%%%%%%%%%%%%%%%%%%%%%%%%%%%%%%%%%%%%
% Draw Commands
%%%%%%%%%%%%%%%%%%%%%%%%%%%%%%%%%%%%%%%%%%%%%%%%%%%%%%%%%%%%%%%%
\csdef{avr@draw@command@0}{} % NOP

%  Set Color
\def\avr@draw@color{0,0,0}
\csdef{avr@draw@command@1}{%
  \ifnum \avr@draw@argc < 3%
    \avr@debug{DRAW: Set Draw COLOR, Not enough arguments}%
  \fi%
  \xdef\avr@draw@color{\avr@draw@stack{0},\avr@draw@stack{1},\avr@draw@stack{2}}%
}

% Small Rectangle
\csdef{avr@draw@command@2}{%
  \ifnum \avr@draw@argc < 2%
    \avr@debug{DRAW: Dot, Not enough arguments (2 required)}%
  \fi%
  \edef\@@X{\avr@draw@stack{0}}%
  \edef\@@Y{\avr@draw@stack{1}}%
  \ifnum \@@X  > \avr@draw@maxx%
     \avr@draw@maxx=\@@X%
  \fi%
  \ifnum \@@Y  > \avr@draw@maxy%
     \avr@draw@maxy=\@@Y%
  \fi%
  \csxdef{avr@draw@\@@X @\@@Y}{\avr@draw@color}%
}


\newcommand{\avrdrawiter}[1]{%
  \avr@draw@x=0%
  \def\avr@iter@yloop{%
    \avr@draw@x=0%
    \ifnum \avr@draw@y > \avr@draw@maxy%
       \let\avr@iter@yloop=\relax%
    \else%
       \let\avr@iter@xloop@rec=\avr@iter@xloop%
       \avr@iter@xloop\relax%
       \advance \avr@draw@y by 1\relax%
    \fi%
    \avr@iter@yloop%
  }%
  \def\avr@iter@xloop{%
    \ifnum \avr@draw@x > \avr@draw@maxx%
       \let\avr@iter@xloop@rec=\relax%
    \else%
       \ifcsdef{avr@draw@\the\avr@draw@x @\the\avr@draw@y}{%
         #1 {\the\avr@draw@x}{\the\avr@draw@y}{\csuse{avr@draw@\the\avr@draw@x @\the\avr@draw@y}}%
        }{% Dot not defined
        }%
       \advance \avr@draw@x by 1\relax%
    \fi%
    \avr@iter@xloop@rec%
  }%
  \avr@iter@yloop%
}

\newcommand{\avrdrawSize}[2]{%
  \avr@count@tmpa=\avr@draw@maxx%
  \advance \avr@count@tmpa by 1\relax%
  \edef#1{\the\avr@count@tmpa}%
  \avr@count@tmpa=\avr@draw@maxy%
  \advance \avr@count@tmpa by 1\relax%
  \edef#2{\the\avr@count@tmpa}%
}

\newwrite\avr@file%

\newcommand{\avrdrawppm}[1]{  
  % Dump to .ppm file
  \immediate\openout\avr@file=#1%
  \immediate\write\avr@file{P3}%
  \avrdrawSize{\x}{\y}%
  \immediate\write\avr@file{\x \space \y \space 255}%
  \def\removecomma##1,##2,##3,{##1 ##2 ##3}%
  \newcommand{\avr@draw@ppm}[3]{%
    \edef\@tempa{##3}%
    \edef\@tempa{\expandafter \removecomma \@tempa,}%
    \expandafter\immediate\expandafter\write\expandafter\avr@file\expandafter{\@tempa}%
  }%
  \avrdrawiter{\avr@draw@ppm}%
  \immediate\closeout\avr@file%
}

%%% Local Variables: 
%%% mode: latex
%%% TeX-master: "avr.tex"
%%% End: 
